\section{Algorithms for Finite Instances}
\label{sec:finiteinst}

We begin our technical presentation by furnishing a lower bound on the sample complexity of algorithms for \QFK.

\subsection{Lower Bound on the Sample-Complexity}
\label{subsec:lbsckmn}

\begin{restatable}{theorem}{thmlbmainthm}[Lower Bound for \QFK]
\label{thm:lbmainthm}
Let $\mathcal{L}$ be an algorithm that solves \QFK. Then, there exists an 
instance $(\A, n, m, k, \epsilon, \delta)$, 
with $0< \epsilon \leq \frac{1}{\sqrt{32}}$, $0 < \delta \leq \frac{e^{-1}}{4}$, 
and $n \geq 2m$, $1 \leq k \leq m$, on which the expected number of pulls 
performed by $\mathcal{L}$ is at least $\frac{1}{18375}. \frac{1}{\epsilon^2}. \frac{n}{m-k+1}\ln\frac{\binom{m}{k - 1}}{4\delta}$.
\end{restatable}

The detailed proof of the theorem is given in Appendix~\ref{app:lowerboundqfk}. 
The proof generalises lower bound proofs for both $(m, m, n)$
~\citep[see Theorem 8]{bib:lucb} and $(1, m, n)$~\citep[see Theorem 3.3]{bib:arcsk2017}.
The core idea in these proofs is to consider two sets of bandit instances,
$\mathcal{I}$ and $\mathcal{I}^{\prime}$, such that over ``short'' trajectories, 
an instance from $\mathcal{I}$ will yield the same reward sequences as a corresponding 
instance from $\mathcal{I}^{\prime}$, with high probability. Thus, any algorithm 
will return the same set of arms for both instances, with high probability. 
However, by construction, no set of arms can be simultaneously correct for both 
instances---implying that a correct algorithm must encounter sufficiently ``long'' 
trajectories. Our main contribution is in the design of 
$\mathcal{I}$ and $\mathcal{I}^{\prime}$ when $k \in \{1, 2, \dots, m\}$ 
(rather than exactly $1$ or $m$) arms have to be returned.

Our algorithms to achieve improved \textit{upper} bounds for \QF and \QFK 
(across bandit instances) follow directly from methods we design for the 
infinite-armed setting in Section~\ref{sec:infinitemab} (see Corollary~\ref{cor:qffromqptighter} 
and Corollary~\ref{cor:qfkfromqpktighter}). In the remainder of this section, we present a fully-sequential algorithm for \QFK whose expected sample complexity varies with the ``hardness'' of the input instance.

% \input{ghalving.tex}
\subsection{An Adaptive Algorithm for Solving \protect\QFK} %\textsc{LUCB}$(k, m)$
\label{sec:adaptive}

% Although there exist instances of \QFK that require within a constant factor of the sample complexity of \GHALVING, the algorithm is likely to be wasteful on ``easy'' instances, in which arms in $\TOPM$ are well-separated from the remaining arms.

%We present an adaptive algorithm, \GLUCB, for solving $(k,m,n)$ and analyse its sample complexity.

Algorithm~\ref{alg:glucb} describes \GLUCB, a fully sequential algorithm, which for $k=1$ has the same
guarantee on sample-complexity as \FF, but empirically appears to be more economical. The algorithm  generalises \LUCB~\cite{bib:lucb}, which solves $(m, m, n)$. 

% However, it differs from
% \LUCB in a subtle manner, due to the very definition of the problem. Like \QF, \QFK
% assumes multiple solutions if $k < m$. On the other hand, it differs from \QF as it
% can solve \SUBSET of size $m \geq 1$ for $k=m$.

% \begin{algorithm}[]
% \small{
% \caption{\GLUCB: Algorithm to select $k$ $(\epsilon, m)$-optimal arms}
% \label{alg:glucb}
% \DontPrintSemicolon% instead
%  \KwIn{$\mathcal{A}$ (\st $|\mathcal{A}| = n$), $k, m, \epsilon, \delta$.}
%  \KwOut{$k$ distinct $(\epsilon,m)$-optimal arms from $\mathcal{A}$.}
%  Pull each arm $a  \in \mathcal{A}$ once. Set $t = n$.\;
%  \Do{$ ucb({l_*^t}, t+1) - lcb({h_*^t}, t+1) > \epsilon.$} { \label{ln:stpkoutofm}
%      $t = t + 1$.\;
% %      $A_1^t = \{a : a' \in \mathcal{A}, \hatp_a = \hatp_{a'}\}$ \st $|A_1^t| = k$.\;
% %      $A_3^t = \{a : a' \in \mathcal{A}, \hatp_a = \hatp_{a'}\}$ \st $|A_3^t| = n-m$.\;
% 	 $A_1^t \defeq $ Set of $k$ arms with the highest empirical means.\;
%      $A_3^t \defeq $ Set of $n-m$ arms with the lowest empirical means.\;
%      $A_2^t \defeq \{\mathcal{A} \setminus (A_1^t \cup A_3^t)\}$.\;
%      $h_*^t = \arg \max_{\{a \in A_1^t\}} lcb(a,t)$.\;
%      $m_*^t = \arg \min_{\{a \in A_2^t\}} u_a^{t}$.\;
%      $l_*^t = \arg \max_{\{a \in A_3^t\}} ucb(a,t)$.\;
%      pull  $h_*^t,  m_*^t, l_*^t$.
%  }
%  \Return $A_1^t$.\;
%  }
% \end{algorithm}

\begin{algorithm}[ht]
\begin{algorithmic}
\small{
 \REQUIRE {$\mathcal{A}$ (\st $|\mathcal{A}| = n$), $k, m, \epsilon, \delta$.}
 \ENSURE {$k$ distinct $(\epsilon,m)$-optimal arms from $\mathcal{A}$.}
 \STATE Pull each arm $a  \in \mathcal{A}$ once. Set $t = n$.
 \WHILE {$ ucb({l_*^t}, t+1) - lcb({h_*^t}, t+1) > \epsilon.$} { \label{ln:stpkoutofm}
     \STATE $t = t + 1$.
     \STATE $A_1^t \defeq $ Set of $k$ arms with the highest empirical means.
     \STATE $A_3^t \defeq $ Set of $n-m$ arms with the lowest empirical means.
     \STATE $A_2^t \defeq \mathcal{A} \setminus (A_1^t \cup A_3^t)$.
     \STATE $h_*^t = \arg \max_{\{a \in A_1^t\}} lcb(a,t)$.
     \STATE $m_*^t = \arg \min_{\{a \in A_2^t\}} u_a^{t}$.
     \STATE $l_*^t = \arg \max_{\{a \in A_3^t\}} ucb(a,t)$.
     \STATE pull  $h_*^t,  m_*^t, l_*^t$.
 }\ENDWHILE
 \STATE Return $A_1^t$.
 }
\end{algorithmic}
\caption{\GLUCB: Algorithm to select $k$ $(\epsilon, m)$-optimal arms}
\label{alg:glucb}
\end{algorithm}


At each round $t$, we partition $\A$ into three subsets. We keep the $k$ arms
with the highest empirical averages in $A_1^t$, the $n-m$ arms with the lowest empirical averages in $A_3^t$,
and the rest in $A_2^t$; ties are broken arbitrarily (uniformly at random in our experiments). At each round we choose
a \emph{contentious} arm from each of these three sets: from 
$A_1^t$ we choose $h_*^t$,
the arm with the lowest lower confidence bound (LCB); from $A_2^t$ the arm which is least pulled is chosen, and called $m_*^t$; from $A_3^t$ we choose $l_*^t$, the arm with the highest
upper confidence bound (UCB). The algorithm stops as soon as the difference between the lower 
confidence bound of $h_*^t$, and the upper confidence bound of $l_*^t$ becomes no larger than 
the tolerance $\epsilon$.

Let $B_1, B_2, B_3$ be corresponding sets based on the true means: that is, subsets of $\mathcal{A}$ such that $B_1 \defeq \{1, 2,\cdots, k\}$,
$B_2 = \{k+1, k+2,\cdots, m\}$ and $B_3=\{m+1, m+2,\cdots, n\}$. For any two arms $a, b \in \mathcal{A}$ we define
$\Delta_{ab} \defeq \mu_a - \mu_b$. For the sake of convenience we slightly overload this notation as
{\footnotesize
\begin{equation}\label{eq:defdelta}
 \Delta_a = \begin{cases}
  \mu_a - \mu_{m+1}\; \text{if}\; a \in B_1\\
  \mu_k - \mu_{m+1}\; \text{if}\; a \in B_2\\
  \mu_m - \mu_a\;\;\;\;\; \text{if}\; a \in B_3.
 \end{cases}
\end{equation}
}
% \begin{table}[H]
% \centering
% \caption{My caption}
% \label{my-label}
% \begin{tabular}{l|lll}
% \hline
%  & $a \in B_1$ & $a \in B_2$ & $a \in B_3$ \\ \hline
% $\Delta_a$ & $\mu_a - \mu_{m+1}$ & $\mu_k - \mu_{m+1}$ & $\mu_m - \mu_a$ \\ \hline
% \end{tabular}
% \end{table}
We note that $\Delta_k = \Delta_{k+1} = \cdots = \Delta_m = \Delta_{m+1}$.
Let $u^*(a,t) \defeq \bceil{\frac{32}{\max\{\Delta_a, \frac{\epsilon}{2}\}^2}\ln\frac{k_1 n t^4}{\delta}}$ for all $a \in \mathcal{A}$, where $k_1=5/4$. 
Now, we define the hardness term as $H_\epsilon = \sum_{a \in \mathcal{A}}\frac{1}{\max\{\Delta_a, \epsilon/2\}^2}$.

\begin{restatable}{theorem}{thmscglucb}[Expected Sample Complexity of \GLUCB]
\label{thm:scglucb}
\GLUCB solves \QFK using an expected sample complexity upper bounded by
$O\left(H_\epsilon \log\frac{H_\epsilon}{\delta}\right)$. 
\end{restatable}
Appendix-A describes the proof in detail. The core argument 
is similar to
that for Algorithm $\F_2$ by \citet{bib:arcsk2017}. However, it subtly differs due to the different strategy for choosing arms since the output set is
not necessarily singleton.
%  Recently, \citet{Jamieson+N:2014} has shown that using a 
% tailored upper bound, \LUCB can be shown to incur an expected sample complexity
% which is within a $O(\log n)$ factor of the lower bound. A similar technique can
% also be adopted here to make a tighter analysis. However, in the interest of
% keeping the proof simple, we keep our analysis restricted in the conventional approach and leave the tighter analysis as a future exercise. 
In practice, one can use
tighter confidence bound calculations (we use KL-divergence based
confidence bounds in our experiments) to get even better sample complexity.
% To analyse the sample complexity, first we define some events, at least
% one of which must occur if the algorithm does not stop at the round $t$.

% \begin{definition}{(\textsc{Probable Events})}
% Let $a, b \in \mathcal{A}$, such that $\mu_a > \mu_b$. During the
% run of the algorithm, any of the following five events may occur:
% $CROSS_a^t \defeq \{ucb(a,t) < \mu_a \vee lcb(a,t) > \mu_a\}$,
% $ErrA(a,b,t)) \defeq \{\hatp_a^t < \hatp_b^t\}$,
% $ErrL(a,b,t) \defeq \{lcb(a,t) < lcb(b,t)\}$,
% $ErrU(a,b,t) \defeq \{ucb(a,t) < ucb(b,t)\}$,
% $ NEEDY_a^t(d) \defeq \{\{lcb(a,t) < \mu_a - d\} \vee \{ucb(a,t) > \mu_a + d\}\}$.
% \end{definition}

% We show that any arm $a$, if sampled sufficiently, that is $u_a^t \geq u^*(a,t)$, 
% then occurrence of any of the \textsc{Probable Events} imply occurrence of $CROSS_a^t$.
% First we show that if  $CROSS_a^t$ does not occur for any $a \in \A$, then occurrence
% of any one of the \textsc{Probable Events} implies the occurrence of $NEEDY_a^t(\cdot)$
% or $NEEDY_b^t(\cdot)$.
% It is important to note that as $m_*^t$ is the least sampled arm in $A_2^t$,
% for any arm $a \in A_2^t$, $NEEDY_a^t \implies NEEDY_{m_*^t}^t$.


% \begin{restatable}{lemma}{lemErrALUN}[Reducing Events To $NEEDY_a^t$]
% \label{lem:ErrALUN}
% To prove that $\{\neg  CROSS_a^t \wedge \neg CROSS_b^t\} \wedge \{ErrA(a,b,t) \vee ErrU(a,b,t) \vee ErrL(a,b,t)\} \implies \{NEEDY_a^t(\frac{\Delta_{ab}}{2}) \vee NEEDY_b^t(\frac{\Delta_{ab}}{2}) \}$.
% \end{restatable}


% We show that given a threshold $d$, if an arm $a$ is sufficiently sample such that $\beta(u_a^t, t, \delta) \leq \frac{d}{2}$, then that $NEEDY_a^t$ infers $CROSS_a^t$.

% \begin{restatable}{lemma}{lemneedycross}
%  \label{lem:needycross}
%   For any $a \in \A$, $\{NEEDY_a^t(d)|\beta(u_a^t, t, \delta) < d/2\} \implies CROSS_a^t$
% \end{restatable}

% By the very definition of confidence bound, at any round $t$, the probability that
% the empirical mean of an arm will lie outside it is very low. In other words, the
% probability that $CROSS_a^t$ occur is very low for all $t$ and $a \in \A$.

% \begin{restatable}{lemma}{lemcross}[Upper bounding the probability of $CROSS_a^t$]
%  \label{lem:cross}
%  $\forall a \in \mathcal{A}$ and $\forall t \geq 0$, $\Pr\{{CROSS_a^t}\}  \leq  \frac{\delta}{knt^4}$. Hence,
%  $P\left[\exists t \geq 0  \wedge \exists a \in \mathcal{A} : {CROSS_a^t} | u_a^t \geq 0  \right] \leq  \frac{\delta}{k_1 t^3}.$
% \end{restatable}
% \begin{proof}
% $\Pr\{{CROSS_a^t}\}$ is upper bounded by using Hoeffding's inequality, and the next statement
% gets proved by taking union bound over all arms and $t$.
% \end{proof}
% Now, recalling the definition of $h_*^t$, and $l_*^t$ from Algorithm~\ref{alg:glucb},
% we present the key logic underlying the analysis of \GLUCB. The idea is to show that
% if the algorithm has not stopped, then one of those \textsc{Probable Events} must have
% occurred. Then using Lemma~\ref{lem:cross}, Lemma~\ref{lem:ErrALUN},
%  and Lemma~\ref{lem:needycross}, 
% we show that beyond a certain number of rounds, the probability that \GLUCB
% will continue is upper bounded is sufficiently small.
% Lastly, using the argument based on pigeon-hole principle, similar to
% \citet[Lemma 5]{bib:shivaramphdthesis}, we establish the upper bound on the 
% sample complexity. The core logic to show that one of the probable events must occur until the algorithm stops is presented below.

% \begin{restatable}{lemma}{lemglucb_cases}
% \label{lem:glucb_cases}
% If at \GLUCB has not stopped at round $t$, then for $a \in \{h_*^t, l_*^t\}$, and $b \in \{h_*^t, l_*^t\} \setminus \{a\}$, one of the \textsc{Probable Events} must have occurred.
% \end{restatable}
% \begin{proof}
% Recalling the definitions of the ground truth sets $B_1, B_2$ and $B_3$,
% as $\{h_*^t, l_*^t\} \in B_1 \cup B_2 \cup B_3$, depending from which sets
% out of these three, $h_*^t$, and $l_*^t$ have come, there are total $3 \times 3 = 9$
% possible cases. 
% The detailed analysis of all the 
% cases is presented in the Appendix~\ref{app:adaptive}.We present the case
% where $h_*^t \in B_2 \wedge l_*^t \in B_2$ in Figure~\ref{fig:logicb2b2}. 
% Any arm in $B_2$ may belong to correct output set, but not mandatory. However, 
% one of the arms in $\{h_*^t, l_*^t\}$ can have mean arbitrarily close to the
% worst arm in $B_1$, while the 
% other one's mean can be arbitrarily close to the best arm in $B_3$.
% On the other hand, as both $h_*^t$, and $l_*^t$
% belong to $B_2$, their means can be arbitrarily close, and hence, even if
% $h_*^t$ and $l_*^t$ are sampled $u^*(h_*^t)$ and $u^*(l_*^t)$ times respectively,
% $ ucb({l_*^t}, t+1) - lcb({h_*^t}, t+1) < \epsilon $ may not hold. 
% However, this will not occur as it will lead some other contradiction. 

% % This special
% % properties of arms in $B_2$ makes this case the most interesting among all.
% % \fbox{
% \begin{figure}
% \framebox[\columnwidth][s]{
% \begin{minipage}[l][][l]{0.45\textwidth}
% \flushleft
%  Suppose $h_*^t \in B_2 \wedge l_*^t \in B_2$ and $\Delta_{h_*^t l_*^t} > 0$.\\
%  Then, $\exists b_1 \in (A_2^t \cup A_3^t)\cap B_1$ and $\exists b_3 \in (A_1^t \cup A_2^t)\cap B_3$.\\
% %   letting $\mu_{h_*^t} > \mu_{l_*^t}$\filler{WRONG!!!}\;
%   \If{$|\Delta_{h_*^t l_*^t}| < \Delta_{h_*^t}/2$}{
%     \If{$\Delta_{b_1 h_*^t} > \Delta_{b_1}/4$}{
%       \If{$b_1 \in A_2^t )\cap B_1$}{
%     $ErrA(b_1, h_*^t, t)$
%       }\Else{
%     $b_1 \in A_3^t \cap B_1$\\
%     $ErrU(b_1, l_*^t, t)$
%       }
%     }\Else{
%       $\Delta_{b_1 h_*^t} \leq \Delta_{b_1}/4$  and hence $\Delta_{l_*^t b_3} \geq \Delta_{l_*^t}/4$\\
%       \If{$b_3 \in A_2^t \cap B_3$}{
%     $ErrA(l_*^t, b_3, t)$  
%       }\Else{
%     $b_3 \in A_1^t \cap B_3$\;
%     $ErrL(h_*^t, b_3, t)$
%       }
%     }
%   }\Else{
%     $|\Delta_{h_*^t l_*^t}| > \Delta_{h_*^t}/2$\\
%     $$NEEDY_{h_*^t}^t (\Delta_{h_*^t}/4) \vee NEEDY_{l_*^t}^t (\Delta_{h_*^t}/4)$$
%   }\par
% \end{minipage}%
% \hspace{2pt}
% \vline
% \hspace{1pt}
% \begin{minipage}[r][][l]{0.45\textwidth}
%  Suppose $h_*^t \in B_2 \wedge l_*^t \in B_2$ and $\Delta_{h_*^t l_*^t} \leq 0$.
%  Then, $\exists b_1 \in (A_2^t \cup A_3^t)\cap B_1$ and $\exists b_3 \in (A_1^t \cup A_2^t)\cap B_3$.\\
% %   letting $\mu_{h_*^t} > \mu_{l_*^t}$\filler{WRONG!!!}\;
%   \If{$|\Delta_{h_*^t l_*^t}| < \Delta_{h_*^t}/2$}{
%     \If{$\Delta_{b_1 l_*^t} > \Delta_{b_1}/4$}{
%       \If{$b_1 \in A_2^t )\cap B_1$}{
%     $ErrA(b_1, h_*^t, t)$
%       }\Else{
%     $b_1 \in A_3^t \cap B_1$\\
%     $ErrU(b_1, l_*^t, t)$
%       }
%     }\Else{
%       $\Delta_{b_1 l_*^t} \leq \Delta_{b_1}/4$ and hence $\Delta_{h_*^t b_3} \geq \Delta_{h_*^t}/4$\\
%       \If{$b_3 \in A_2^t \cap B_3$}{
%     $ErrA(l_*^t, b_3, t)$  
%       }\Else{
%     $b_3 \in A_1^t \cap B_3$\\
%     $ErrL(h_*^t, b_3, t)$
%       }
%     }
%   }\Else{
%     $|\Delta_{h_*^t l_*^t}| > \Delta_{h_*^t}/2$\\
%     $$NEEDY_{h_*^t}^t (\Delta_{h_*^t}/4) \vee NEEDY_{l_*^t}^t (\Delta_{h_*^t}/4)$$
%   }
% \end{minipage}
% }
% \caption{Shows that if \GLUCB does not stop and both $h_*^t$, $l_*^t \in B_2$,
% then one of the \textsc{Probable Events} must have occurred. For the analysis of 
% all the cases vide Appendix~\ref{app:adaptive}.}
% \label{fig:logicb2b2}
% \end{figure}
% \end{proof}

% \begin{restatable}{corollary}{corgenubscglucb}
% \label{cor:gen_ubsc_glucb}
% For any arm  $a \A$, our definition of $\Delta_a$ coincides with the one
% defined by \citet{bib:arcsk2017} for $k=1$, and for $k=m$ it coincides with
% the one by \citet{bib:lucb}. As The hardness term $H_\epsilon$ is the same
% function of $\Delta_a$ and $\epsilon$, the upper bound on the sample complexity of
% \GLUCB matches with those for $\F_2$ and \LUCB for $k=1$, and $k=m$ respectively.
% \end{restatable}

% \GLUCB being an adaptive algorithm the sample complexity is dependent on the relative differences
% of means of the arms. Next, we are going to present a non-adaptive algorithm that solves \QFK
% with a number of samples that matches the lower bound up to a constant factor.
%  We two algorithms that solve \QFK: one with optimal worst-case sample complexity, and the other
Next, we are going to consider infinite-armed bandit instances, and present the algorithms to solve them.